\documentclass[11pt,a4paper,oneside]{article}
\usepackage[utf8]{inputenc}
\usepackage[english,russian]{babel}
\usepackage{amssymb}
%\usepackage{amsmath}
%\usepackage{mathabx}
\usepackage{stmaryrd}
\usepackage[left=2cm,right=2cm,top=2cm,bottom=2cm,bindingoffset=0cm]{geometry}
\usepackage{bnf}
\newcommand{\lit}[1]{\mbox{`\texttt{#1}'}}
\newcommand{\ntm}[1]{<\mbox{#1}>}
\begin{document}


\begin{center}
\begin{Large}{\bfseries Вопросы к коллоквиуму по курсу <<Математическая логика>>}\end{Large}\\
\vspace{1mm}
\begin{small} ИТМО, группы M3234..M3239\end{small}\\
\small 18 и 19 апреля 2018 г.
\end{center}

\begin{itemize}
\item Топология: топологическое пространство, база топологического пространства, 
открытое и замкнутое множество, внутренность и замыкание множества, топология стрелки,
дискретная топология, топология на частично упорядоченном множестве,
индуцированная топология на подпространстве, связность.
\item Исчисление высказываний: высказывание, аксиома,
схема аксиом, правило Modus Ponens, доказательство,
вывод из гипотез, доказуемость, 
множество истинностных значений, модель (оценка переменных), оценка высказывания, общезначимость, 
корректность, полнота, формулировка теоремы о дедукции
\item Интуиционистское исчисление высказываний: 
закон исключённого третьего, BHK-интерпретация логических связок, теорема Гливенко (формулировка),
решётка, дистрибутивная решётка, импликативная решётка, 
алгебра Гейтинга, булева алгебра, гомоморфизм алгебр Гейтинга,
Гёделева алгебра, операция $\Gamma(A)$, алгебра Линденбаума,
модель Крипке, вложение моделей Крипке в алгебры Гейтинга,
формулировка свойства дизъюнктивности и.и.в, формулировка свойства нетабличности и.и.в.
\item Исчисление предикатов:
предикатные и функциональные символы, константы и пропозициональные переменные,
свободные и связанные вхождения предметных переменных в формулу, 
свобода для подстановки, два правила для кванторов, две аксиомы для кванторов,
модель в исчислении предикатов, полное множество (бескванторных) формул, 
модель для формулы, теорема Гёделя о полноте исчисления предикатов (формулировка).
\end{itemize}

\end{document}