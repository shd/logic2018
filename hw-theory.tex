\documentclass[10pt,a4paper,oneside]{article}
\usepackage[utf8]{inputenc}
\usepackage[english,russian]{babel}
\usepackage{amsmath}
\usepackage{amsthm}
\usepackage{amssymb}
\usepackage{enumerate}
\usepackage{stmaryrd}
\usepackage[left=2cm,right=2cm,top=2cm,bottom=2cm,bindingoffset=0cm]{geometry}
\usepackage{proof}
\newcommand{\gq}[1]{\texttt{<<}#1\texttt{>>}}
\newcommand{\ogq}[1]{\overline{\texttt{<<}#1\texttt{>>}}}
\begin{document}

\begin{center}{\Large\textsc{\textbf{Теоретические (``малые'') домашние задания}}}\\
             \it Математическая логика, ИТМО, М3234-М3239, весна 2018 года\end{center}

\section*{Домашнее задание №1: <<знакомство с исчислением высказываний>>}

Докажите при любых подстановках метапеременных $\alpha$, $\beta$ и $\gamma$:
\begin{enumerate}
\item $\vdash\alpha\&\beta\rightarrow\beta\&\alpha$
\item $\vdash\alpha \rightarrow \neg\neg \alpha$
\item $\vdash\alpha\&(\beta\vee\gamma) \rightarrow (\alpha\vee\beta)\&(\alpha\vee\gamma)$
\item $\vdash\neg(\alpha\&\beta) \rightarrow \neg\alpha\vee\neg\beta$
\item $\vdash(\alpha\rightarrow\beta)\rightarrow(\neg\beta\rightarrow\neg\alpha)$
\end{enumerate}

\section*{Домашнее задание №2: <<теорема о полноте исчисления высказываний>>}

В данном домашнем задании вам будет предложено доказать несколько
важных лемм, используемых в теореме о полноте исчисления
высказываний. Подробнее с этой теоремой можно ознакомиться в конспекте курса,
глава 5. В решениях можно пользоваться всем ранее доказанным
на парах и в других домашних заданиях.

\begin{enumerate}
\item Докажите при любых значениях метапеременных $\alpha$, $\beta$:
\begin{enumerate}
\item $\alpha,\beta \vdash \alpha\&\beta$
\item $\neg\alpha,\beta \vdash \neg(\alpha\&\beta)$
\item $\alpha,\neg\beta \vdash \neg(\alpha\&\beta)$
\item $\neg\alpha,\neg\beta \vdash \neg(\alpha\&\beta)$
\item $\alpha,\beta \vdash \alpha\vee\beta$
\item $\neg\alpha,\beta \vdash \alpha\vee\beta$
\item $\alpha,\neg\beta \vdash \alpha\vee\beta$
\item $\neg\alpha,\neg\beta \vdash \neg(\alpha\vee\beta)$
\item $\alpha,\beta \vdash \alpha\rightarrow\beta$
\item $\alpha,\neg\beta \vdash \neg(\alpha\rightarrow\beta)$
\item $\neg\alpha,\beta \vdash \alpha\rightarrow\beta$
\item $\neg\alpha,\neg\beta \vdash \alpha\rightarrow\beta$
\item $\neg\alpha \vdash \neg\alpha$
\item $\alpha \vdash \neg\neg\alpha$
\end{enumerate}

\item Докажите, что при любых значениях метапеременной $\alpha$ 
справедливо $\vdash \alpha\vee\neg\alpha$

\item Докажите, что при любых списках формул $\Gamma$ и $\Delta$ и при любых
значениях метапеременных $\gamma$,$\delta$,$\zeta$
если $\Gamma \vdash \gamma$, $\Delta \vdash \delta$ и $\gamma,\delta\vdash\zeta$,
то $\Gamma,\Delta\vdash\zeta$

\item Докажите, что если $\Gamma, \rho \vdash \alpha$ и $\Gamma, \neg\rho \vdash \alpha$,
то $\Gamma \vdash \alpha$
\end{enumerate}

\section*{Домашнее задание №3: <<интуиционистское исчисление высказываний>>}

Введём обозначение: нижним индексом у <<турникета>> будем указывать логику, в которой 
проводится доказательство. Если высказывание $\alpha$ доказуемо в интуиционистской логике,
будем писать $\vdash_\texttt{И}\alpha$, если в классической --- $\vdash_\texttt{К}\alpha$.

\begin{enumerate}

\item Напомним, как на лекции определялась оценка высказываний интуиционистского 
исчисления на топологическом пространстве $\langle X, \Omega \rangle$:

\begin{tabular}{l}\\
$\llbracket \alpha \& \beta \rrbracket = \llbracket \alpha \rrbracket \cap \llbracket \beta \rrbracket$\\
$\llbracket \alpha \vee \beta \rrbracket = \llbracket \alpha \rrbracket \cup \llbracket \beta \rrbracket$\\
$\llbracket \alpha \rightarrow \beta \rrbracket = \mathrm{int}(\mathrm{c}(\llbracket \alpha \rrbracket) \cup \llbracket \beta \rrbracket)$\\
$\llbracket \neg \alpha \rrbracket = \mathrm{int}(\mathrm{c}(\llbracket \alpha \rrbracket))$
\end{tabular}

Также, положим, что высказывание $\alpha$ истинно, если $\llbracket\alpha\rrbracket = X$
(т.е. любое доказуемое высказывание неизбежно имеет оценку, равную всему пространству).
Докажите, что так опеределённая оценка корректна.

\item Докажите теорему Гливенко: $\vdash_\texttt{К}\alpha$ тогда и только тогда, когда
$\vdash_\texttt{И}\neg\neg\alpha$. Чтобы это сделать, сперва докажите три вспомогательных
утверждения:
\begin{enumerate}
\item $\vdash_\texttt{И}\neg\neg\alpha$, если $\alpha$ --- некоторая аксиома интуиционистского
исчисления высказываний.
\item При любом $\alpha$ выполнено $\vdash_\texttt{И}\neg\neg(\neg\neg\alpha \rightarrow \alpha)$
\item При любых $\alpha$ и $\beta$, если $\vdash_\texttt{И}\neg\neg\alpha$ и
$\vdash_\texttt{И}\neg\neg(\alpha\rightarrow\beta)$, то $\vdash_\texttt{И}\neg\neg\beta$
\end{enumerate}

\item Покажите с помощью опровергающего примера, что в интуиционистской логике не выполнено:
\begin{enumerate}
\item $\vdash_\texttt{И}\neg\neg P\rightarrow P$
\item $\vdash_\texttt{И}((P\rightarrow Q)\rightarrow P)\rightarrow P$
 (<<закон Пирса>>)
\end{enumerate}

\item (Задача Куратовского) Будем применять к множеству в некоторой топологии различные 
последовательности операций $\mathrm{int}$ и $\mathrm{cl}$ и смотреть на получившиеся
результаты. Некоторые множества будут
совпадать: скажем, всегда $\mathrm{int}A = \mathrm{int}(\mathrm{int}A)$, а некоторые будут
различны. Сколько вообще возможно получить различных множеств таким способом?

\end{enumerate}

\end{document}